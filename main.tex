\documentclass{article}

% Preamble
\usepackage[utf8]{inputenc}
\usepackage{amsmath, amsfonts, amsthm, amssymb}


\theoremstyle{definition}
\newtheorem{definition}{Definition}
\newtheorem{thm}{Theorem}
\newtheorem{prop}{Proposition}
\newtheorem{example}{Example}
\newtheorem{lemma}{Lemma}
\newtheorem{cor}{Corollary}
\newtheorem*{remark}{Remark}




\author{Your name}
% end preamble

\title{Your title}

\begin{document}

\maketitle



\section{Introduction}
introduction

\section{Data structures}

\begin{definition}[Face]
    Something is a face if it satisfies:
    \begin{itemize}
        \item this item
        \item this item
    \end{itemize}
\end{definition}


\section{Let the math begin}
We can write math
\begin{align}
    x^{2} + y^{2} = c^{2}
\end{align}


\begin{definition}[collection]
    Elements $f_1, \dots, f_n$ is said to form a collection
\end{definition}


\begin{prop}[]
    Cool result
    \begin{proof}
        Left as exercise
    \end{proof}
\end{prop}

\end{document}